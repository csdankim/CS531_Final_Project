\chapter{Conclusions}
\label{results}
The purpose of this thesis was to measure the performance of MCTS in Sokoban. We developed various optimizations for MCTS and verified their effectiveness on Sokoban and Samegame. RAVE obtained poor results in both domains and we concluded that neither Sokoban nor Samegame fit the AMAF criteria, according to which all moves performed during a simulation can be considered as if they were played at the beginning. UCB1-Tuned obtained good results on Samegame, improving the previous score obtained with SP-MCTS. Node Recycling obtained poor results in both games in terms of score and number of solved levels, although its major contribution concerns memory usage. SP-MCTS performed better than UCB1-Tuned but worse than UCT on Sokoban. Node Elimination and Cycles Avoidance provided a considerable improvement in Sokoban, especially when used together, since they were able to greatly increase the number of nodes added to the tree and, as a consequence, the number of solved levels. In Samegame, Node Elimination obtained poor performance, mainly because the aim was to maximize the score instead of finding a single solution, meaning that revisiting terminal nodes could be beneficial to the accuracy of the action-value function estimate. As a comparison, IDA* performed very poorly in Samegame, while in Sokoban it obtained the best results. A brief analysis of Sokoban and the chosen heuristic rewards led us to believe that the poor performance obtained by MCTS may be caused by the fact that, while typically MCTS is used on games where at the end of the simulation the outcome is known, in Sokoban at the end of a simulation we needed to provide an evaluation of the distance from the solution. Since computing this evaluation would be as difficult as solving the level, we could only provide a heuristic estimate. IDA*, on the other hand, is designed to use heuristics to guide the search and prune the tree. According to the results obtained with the tested configurations, MCTS doesn't appear to be successful in Sokoban.

\section{Future research}
During our experiments on Sokoban we noticed that the Microban suite was split between very easy levels, which could be solved with very few iterations and very hard levels, which couldn't be solved within reasonable time. As shown by Figures \ref{fig:sokobanIDAComplexity} and \ref{fig:sokobanMCTSComplexity} this was true for both MCTS and IDA*, which means that it was a characteristic of the levels and not of the specific algorithm. A more precise analysis on the effects of the optimizations could be performed by building a test set with levels of gradually increasing difficulty, in order to obtain more accurate indications on the effect of each parameter and optimization.

\medskip\noindent
On the subject of improving MCTS performance on Sokoban, better results could be obtained with more complex reward functions, which make use of domain knowledge to improve the accuracy of the estimated action values.

\medskip\noindent
One final interesting research topic can be the application of the newly proposed Node Elimination and Cycles Avoidance in other domains. Cycles Avoidance in the Avoid Cycles variant should be effective in all games where cycles frequently occur, while our results suggest that Node Elimination could provide benefits in those domains where terminal nodes appear early in the game tree and the goal is not to maximize a score, but to find a single solution.
