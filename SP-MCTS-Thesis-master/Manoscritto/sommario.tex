\newpage
\chapter*{Sommario}
\textit{Monte Carlo Tree Search} (MCTS) è uno degli algoritmi più utilizzati nel campo dell'intelligenza artificiale applicata ai giochi da tavolo e ai giochi di carte.
Lo scopo di questa tesi è di valutare le prestazioni dell'algoritmo MCTS applicato al puzzle game Sokoban, confrontandolo con un altro algoritmo ben noto, \textit{Iterative Deepening A*}, che ha dimostrato di avere un discreto successo in questo puzzle game.
In questa tesi applichiamo MCTS e IDA* a Sokoban e Samegame, un altro puzzle game. Sviluppiamo anche una serie di ottimizzazioni conosciute per MCTS e IDA* e ne presentiamo alcune nuove. Infine, valutiamo gli effetti delle ottimizzazioni su entrambi i domini. I nostri risultati mostrano che in Samegame la formula UCB1-Tuned ottiene prestazioni migliori rispetto a SP-MCTS, una versione single player di MCTS che ha ottenuto buoni risultati in quel dominio in passato. In Sokoban, la migliore configurazione del MCTS usa l'UCT standard con l'aggiunta delle ottimizzazioni proposte denominate Node Elimination e Cycles Avoidance, che portano ad un drastico aumento del numero di livelli risolti dall'algoritmo MCTS in Sokoban. Nonostante ciò, anche con una serie di miglioramenti che possono essere trovati in letteratura, che sono stati ampiamente utilizzati e hanno raggiunto risultati di successo, l'algoritmo MCTS non ha potuto eguagliare le prestazioni di IDA* in termini di numero di livelli risolti. Per questo motivo IDA* rimane ancora il miglior algoritmo per Sokoban.