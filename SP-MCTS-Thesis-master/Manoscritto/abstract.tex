\newpage
\chapter*{Abstract}
\textit{Monte Carlo Tree Search} (MCTS) is one of the most used algorithm in the field of Artificial Intelligence applied to board and card games.
The aim of this thesis is to evaluate the performance of MCTS algorithm applied to the puzzle game Sokoban, comparing it to another well known algorithm, \textit{Iterative Deepening A*}, which has been proven to be quite successful in this puzzle game.
In this thesis we apply MCTS and IDA* to Sokoban and Samegame, another puzzle game. We also develop a series of known optimizations to MCTS and IDA* and present some new ones. Finally, we evaluate the effects of the optimizations on both domains. Our results show that in Samegame the UCB1-Tuned formula performs better than SP-MCTS, a single player version of MCTS that obtained good results in that domain in the past. In Sokoban, the best MCTS configuration uses the standard UCT with the addition of the proposed optimizations called Node Elimination and Cycles Avoidance, which lead to a drastic increase in the number of levels solved. Despite this, even with a set of enhancements that can be found in the literature, which have been widely used and have achieved successful results, the MCTS algorithm could not match IDA* performance in terms of number of solved levels. For this reason IDA* still remains the best algorithm for Sokoban.